\documentclass [a4paper,11pt]{article}
\usepackage{booktabs}
\usepackage{zhfontcfg}
\usepackage{multirow}
\usepackage[margin=1in]{geometry}


\title{补充性规格说明}
\date{\today}
\author{LetsGo安卓应用开发小组}

\begin{document}
	
\maketitle
\section*{修订历史}

\begin{table}[!hbp]
\centering

\begin{tabular*}{\textwidth}{c|c|c|c}
\hline
\rule{0pt}{0.8cm}
版~本 & 日~期 & 描~述 & 作~者\\
\hline
\rule{0pt}{0.6cm}
初始草案 & 2014年5月11日 & 第一个草案。主要进行细化以及精化。 & 王枫荻\\
\hline
\rule{0pt}{0.6cm}
 &  &  & \\
\hline
\end{tabular*}

\end{table}

\section*{简介}
本文档记录了LetsGo安卓应用除去已经在用例文档中描述的需求。
\section*{功能性}
\subsection*{1.统一的计划管理}
在整个系统中,采用相同的任务列表。不论是对于查看计划、记录跑步内容还是设定计划,都是采用统一的单例列表进行管理。
\subsection*{2.可插拔规则}
允许系统进行额外定制。
\section*{可用性}
系统提供易于理解的用户文档,帮助用户更好地使用系统。在系统中,如果出现异常错误,会以十分友好的界面提示用户,使用户正确操作,达到用户需求。
\section*{可靠性}
\subsection*{1.可恢复性}
系统设计合理,架构明确,代码健壮,故障率较低。并且对用户可能的异常操作做了充分的考虑,使得异常出现频率较低,而且即使出现异常,也能很好地恢复。
\subsection*{2.性能}
系统轻量化,响应时间较短。而且系统所需硬件资源较少,能给用户好的体验。
\section*{可支持性}
系统基于较流行的安卓平台,有好的支持。此外以后系统会在之前版本的基础上提供更新,修复漏洞,不断完善。

\section*{接口}
\subsection*{1.硬件接口}
触摸屏、GPS、时钟
\subsection*{2.软件接口}
安卓操作系统
  
\end{document}
