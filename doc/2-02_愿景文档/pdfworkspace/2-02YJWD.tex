\documentclass [a4paper,11pt]{article}
\usepackage{booktabs}
\usepackage{zhfontcfg}
\usepackage{multirow}
\usepackage[margin=1in]{geometry}


\title{愿景文档}
\date{\today}
\author{LetsGo安卓应用开发小组}

\begin{document}
	
\maketitle
\section*{修订历史}

\begin{table}[!hbp]
\centering

\begin{tabular*}{\textwidth}{c|c|c|c}
\hline
\rule{0pt}{0.8cm}
版~本 & 日~期 & 描~述 & 作~者\\
\hline
\rule{0pt}{0.6cm}
初始草案 & 2014年5月11日 & 第一个草案。主要在细化阶段中产生设想。 & 马锐骕骏\\
\hline
\rule{0pt}{0.6cm}
 &  &  & \\
\hline
\end{tabular*}

\end{table}

\section*{简介}
随着我国市场经济的发展人们的生活水平逐渐提高,人们有了“想花钱买健康”的意识,健康生活理念已经越来越深入人心。但是健身俱乐部在我国发展的时间还比较短,俱乐部的水平参差不齐,所以健身的软件一定会有非常好的发展潜力和市场。
\section*{定位}
\subsection*{1.商业机遇}
随着人们的健康意识不断增强,健康的管理成为一个及其重要的领域,在这个领域中充斥这很多机会,如健身只能硬件、健康管理软件等;
\subsection*{2.问题综述}
本地化的健康管理软件,随着需求的不断增加,有可能会扩展分享、在线交友、在线交流等网络功能;
\subsection*{3.选择和竞争}
为了增快软件的交互,暂时不要网络功能,因为网络在不同地点不同手机不同套餐快慢是不确定的。所以第一代的版本是针对本地化的个人使用的版本。这也是最大的优势也可能是最大的劣势。
\end{document}
